\documentclass[runningheads,a4paper]{llncs}

\usepackage[american]{babel}

%better font, similar to the default springer font
%cfr-lm is preferred over lmodern. Reasoning at http://tex.stackexchange.com/a/247543/9075
\usepackage[%
rm={oldstyle=false,proportional=true},%
sf={oldstyle=false,proportional=true},%
tt={oldstyle=false,proportional=true,variable=true},%
qt=false%
]{cfr-lm}
%
%if more space is needed, exchange cfr-lm by mathptmx
%\usepackage{mathptmx}

%Enable direct input of umlauts in the bibliography
\usepackage[utf8]{inputenc}

\usepackage{graphicx}

%extended enumerate, such as \begin{compactenum}
\usepackage{paralist}

%put figures inside a text
%\usepackage{picins}
%use
%\piccaptioninside
%\piccaption{...}
%\parpic[r]{\includegraphics ...}
%Text...

%Sorts the citations in the brackets
%\usepackage{cite}

\usepackage[T1]{fontenc}

%for demonstration purposes only
\usepackage[math]{blindtext}

%for easy quotations: \enquote{text}
\usepackage{csquotes}

%enable margin kerning
\usepackage{microtype}

%tweak \url{...}
\usepackage{url}
%nicer // - solution by http://tex.stackexchange.com/a/98470/9075
%\makeatletter
%\def\Url@twoslashes{\mathchar`\/\@ifnextchar/{\kern-.2em}{}}
%\g@addto@macro\UrlSpecials{\do\/{\Url@twoslashes}}
%\makeatother
\urlstyle{same}
%improve wrapping of URLs - hint by http://tex.stackexchange.com/a/10419/9075
\makeatletter
\g@addto@macro{\UrlBreaks}{\UrlOrds}
\makeatother

%diagonal lines in a table - http://tex.stackexchange.com/questions/17745/diagonal-lines-in-table-cell
%slashbox is not available in texlive (due to licensing) and also gives bad results. This, we use diagbox
%\usepackage{diagbox}

%required for pdfcomment later
\usepackage{xcolor}

%%%
%biblatex statt bibtex
\usepackage[
  backend       = biber, %minalphanames only works with biber backend
  sortcites     = true,
  bibstyle      = alphabetic,
  citestyle     = alphabetic,
  firstinits    = true,
  useprefix     = false, %"von, van, etc." will be printed, too. See below.
  minnames      = 1,
  minalphanames = 3,
  maxalphanames = 4,
  maxbibnames   = 99,
  maxcitenames  = 3,
  doi           = false, %source: http://tex.stackexchange.com/a/23118/9075
  isbn          = false, %source: http://tex.stackexchange.com/a/23118/9075
  backref       = false]{biblatex}

%paper.bib is the source of our references
\bibliography{paper}

%Do not put "vd" in the label, but put it at "\citeauthor"
%Source: http://tex.stackexchange.com/a/30277/9075
\makeatletter
\AtBeginDocument{\toggletrue{blx@useprefix}}
\AtBeginBibliography{\togglefalse{blx@useprefix}}

%Thin spaces between initials
%http://tex.stackexchange.com/a/11083/9075
\renewrobustcmd*{\bibinitdelim}{\,}

%Keep first and last name together in the bibliography
%http://tex.stackexchange.com/a/196192/9075
\renewcommand*\bibnamedelimc{\addnbspace}
\renewcommand*\bibnamedelimd{\addnbspace}

%Replace last "and" by comma in bibliography
%See http://tex.stackexchange.com/a/41532/9075
%\AtBeginBibliography{%
%  \renewcommand*{\finalnamedelim}{\addcomma\space}%
%}

\DefineBibliographyStrings{ngerman}{
  backrefpage  = {zitiert auf S\adddot},
  backrefpages = {zitiert auf S\adddot},
  andothers    = {et\ \addabbrvspace al\adddot},
  %Tipp von http://www.mrunix.de/forums/showthread.php?64665-biblatex-Kann-%DCberschrift-vom-Inhaltsverzeichnis-nicht-%E4ndern&p=293656&viewfull=1#post293656
  bibliography = {Literaturverzeichnis}
}

%enable hyperlinked author names when using \citeauthor
%source: http://tex.stackexchange.com/a/75916/9075
\DeclareCiteCommand{\citeauthor}
  {\boolfalse{citetracker}%
   \boolfalse{pagetracker}%
   \usebibmacro{prenote}}
  {\ifciteindex
     {\indexnames{labelname}}
     {}%
   \printtext[bibhyperref]{\printnames{labelname}}}
  {\multicitedelim}
  {\usebibmacro{postnote}}

%natbib compatibility
\newcommand{\citep}[1]{\cite{#1}}
\newcommand{\citet}[1]{\citeauthor{#1} \cite{#1}}
%Beginning of sentence - analogous to cleveref - important for names such as "zur Muehlen"
\newcommand{\Citep}[1]{\cite{#1}}
\newcommand{\Citet}[1]{\Citeauthor{#1} \cite{#1}}
%%%

% new packages BEFORE hyperref
% See also http://tex.stackexchange.com/questions/1863/which-packages-should-be-loaded-after-hyperref-instead-of-before

%enable hyperref without colors and without bookmarks
\usepackage[
%pdfauthor={},
%pdfsubject={},
%pdftitle={},
%pdfkeywords={},
bookmarks=false,
breaklinks=true,
colorlinks=true,
linkcolor=black,
citecolor=black,
urlcolor=black,
%pdfstartpage=19,
pdfpagelayout=SinglePage,
pdfstartview=Fit
]{hyperref}
%enables correct jumping to figures when referencing
\usepackage[all]{hypcap}

%enable nice comments
\usepackage{pdfcomment}
\newcommand{\commentontext}[2]{\colorbox{yellow!60}{#1}\pdfcomment[color={0.234 0.867 0.211},hoffset=-6pt,voffset=10pt,opacity=0.5]{#2}}
\newcommand{\commentatside}[1]{\pdfcomment[color={0.045 0.278 0.643},icon=Note]{#1}}

%compatibality with TODO package
\newcommand{\todo}[1]{\commentatside{#1}}

%enable \cref{...} and \Cref{...} instead of \ref: Type of reference included in the link
\usepackage[capitalise,nameinlink]{cleveref}
%Nice formats for \cref
\crefname{section}{Sect.}{Sect.}
\Crefname{section}{Section}{Sections}

\usepackage{xspace}
%\newcommand{\eg}{e.\,g.\xspace}
%\newcommand{\ie}{i.\,e.\xspace}
\newcommand{\eg}{e.\,g.,\ }
\newcommand{\ie}{i.\,e.,\ }

%introduce \powerset - hint by http://matheplanet.com/matheplanet/nuke/html/viewtopic.php?topic=136492&post_id=997377
\DeclareFontFamily{U}{MnSymbolC}{}
\DeclareSymbolFont{MnSyC}{U}{MnSymbolC}{m}{n}
\DeclareFontShape{U}{MnSymbolC}{m}{n}{
    <-6>  MnSymbolC5
   <6-7>  MnSymbolC6
   <7-8>  MnSymbolC7
   <8-9>  MnSymbolC8
   <9-10> MnSymbolC9
  <10-12> MnSymbolC10
  <12->   MnSymbolC12%
}{}
\DeclareMathSymbol{\powerset}{\mathord}{MnSyC}{180}

% correct bad hyphenation here
\hyphenation{op-tical net-works semi-conduc-tor}

\begin{document}

%Works on MiKTeX only
%hint by http://goemonx.blogspot.de/2012/01/pdflatex-ligaturen-und-copynpaste.html
%also http://tex.stackexchange.com/questions/4397/make-ligatures-in-linux-libertine-copyable-and-searchable
%This allows a copy'n'paste of the text from the paper
\input glyphtounicode.tex
\pdfgentounicode=1

\title{Paper Title}
%If Title is too long, use \titlerunning
%\titlerunning{Short Title}

%Single insitute
\author{Firstname Lastname \and Firstname Lastname}
%If there are too many authors, use \authorrunning
%\authorrunning{First Author et al.}
\institute{Institute}

%Multiple insitutes
%Currently disabled
%
\iffalse
%Multiple institutes are typeset as follows:
\author{Firstname Lastname\inst{1} \and Firstname Lastname\inst{2} }
%If there are too many authors, use \authorrunning
%\authorrunning{First Author et al.}

\institute{
Insitute 1\\
\email{...}\and
Insitute 2\\
\email{...}
}
\fi
			
\maketitle

\begin{abstract}
Abstract goes here
\end{abstract}

\keywords{...}

%%%%%%%%%%%%%%%%%%%%%%%%%%%%%%%%%%%%%%%%%%%%%%%%%%%%%%%%%%%%%%%%%%%%%%%%%%%%%%%
\section{Introduction}\label{sec:intro}
%%%%%%%%%%%%%%%%%%%%%%%%%%%%%%%%%%%%%%%%%%%%%%%%%%%%%%%%%%%%%%%%%%%%%%%%%%%%%%%
\blindtext\todo{Refine me}

Winery~\cite{Kopp2013_Winery} is graphical \commentontext{modeling}{modeling with one \enquote{l}, because of AE} tool.

This is only a demonstration:
We cite two books \cite{vermesan2013internet, goswami_internet_2012},
two articles \cite{Petcu2014_MultiCloud, subramoni_design_2008},
one dissertation \cite{fielding_architectural_2000},
one workshop paper \cite{Kopp2013_Winery},
and two web sites \cite{piper_choosing,ApacheODE}.


\begin{figure}
Simple Figure
\caption{Simple Figure}
\label{fig:simple}
\end{figure}

\begin{table}
\caption{Simple Table}

\label{tab:simple}
Simple Table
\end{table}

cref Demonstration: Cref at beginning of sentence, cref in all other cases.

\Cref{fig:simple} shows a simple fact, although \cref{fig:simple} could also show something else.

\Cref{tab:simple} shows a simple fact, although \cref{tab:simple} could also show something else.

\Cref{sec:intro} shows a simple fact, although \cref{sec:intro} could also show something else.

Brackets work as designed:
<test>

The symbol for powerset is now correct: $\powerset$ and not a Weierstrass p ($\wp$).

\begin{inparaenum}
\item All these items...
\item ...appear in one line
\item This is enabled by the paralist package.
\end{inparaenum}

``something in quotes'' using plain tex or use \enquote{the enquote command}.

\section{Conclusion and Outlook}

\subsubsection*{Acknowledgments}
...

In the bibliography, use \texttt{\textbackslash textsuperscript} for ``st'', ``nd'', ...:
E.g., \enquote{The 2\textsuperscript{nd} conference on examples}.
When you use \href{http://www.jabref.org}{JabRef}, you can use the clean up command to achieve that.


% We do NOT use splncs03, since we switched to biblatex
\printbibliography

All links were last followed on October 5, 2014.

\end{document}
