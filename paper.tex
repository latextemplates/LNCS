% This template has been tested with LLNCS DOCUMENT CLASS -- version 2.20 (24-JUN-2015)

%"runningheads" enables:
%  - page number on page 2 onwards
%  - title/authors on even/odd pages
%This is good for other readers to enable proper archiving among other papers and pointing to
%content. Even if the title page states the title, when printed and stored in a folder, when
%blindly opening the folder, one could hit not the title page, but an arbitrary page. Therefore,
%it is good to have title printed on the pages, too.
\documentclass[runningheads,a4paper]{llncs}[2015/06/24]

%cmap has to be loaded before any font package (such as cfr-lm)
\usepackage{cmap}
\usepackage[T1]{fontenc}

\usepackage{graphicx}

% für neue deutsche Rechtschreibung
\usepackage[english,ngerman]{babel}

% für englische Rechtschreibung
%Even though `american`, `english` and `USenglish` are synonyms for babel package (according to https://tex.stackexchange.com/questions/12775/babel-english-american-usenglish), the llncs document class is prepared to avoid the overriding of certain names (such as "Abstract." -> "Abstract" or "Fig." -> "Figure") when using `english`, but not when using the other 2.
%english has to go last to set it as default language
%\usepackage[ngerman,english]{babel}

%Eingabeformat UTF-8
\usepackage[utf8]{inputenc}

%Hint by http://tex.stackexchange.com/a/321066/9075 -> enable "= as dashes
\addto\extrasenglish{\languageshorthands{ngerman}\useshorthands{"}}

%cfr-lm is preferred over lmodern. Reasoning at http://tex.stackexchange.com/a/247543/9075
\usepackage[%
rm={oldstyle=false,proportional=true},%
sf={oldstyle=false,proportional=true},%
tt={oldstyle=false,proportional=true,variable=true},%
qt=false%
]{cfr-lm}
%
%if more space is needed, exchange cfr-lm by mathptmx

\graphicspath{{graphics/}}

%Tweaks by IPVS/AS
\usepackage{lncs_as}

%for demonstration purposes only
\usepackage[math]{blindtext}

%Sorts the citations in the brackets
%It also allows \cite{refa, refb}. Otherwise, the document does not compile.
%  Error message: "White space in argument"
\usepackage{cite}


%% If you need packages for other papers,
%% START COPYING HERE
%% COPY ALSO cmap and fontenc from lines 10 to 12

%extended enumerate, such as \begin{compactenum}
\usepackage{paralist}

%put figures inside a text
%\usepackage{picins}
%use
%\piccaptioninside
%\piccaption{...}
%\parpic[r]{\includegraphics ...}
%Text...

%for easy quotations: \enquote{text}
\usepackage{csquotes}

%enable margin kerning
\usepackage{microtype}

%tweak \url{...}
\usepackage{url}
%\urlstyle{same}
%improve wrapping of URLs - hint by http://tex.stackexchange.com/a/10419/9075
\makeatletter
\g@addto@macro{\UrlBreaks}{\UrlOrds}
\makeatother
%nicer // - solution by http://tex.stackexchange.com/a/98470/9075
%DO NOT ACTIVATE -> prevents line breaks
%\makeatletter
%\def\Url@twoslashes{\mathchar`\/\@ifnextchar/{\kern-.2em}{}}
%\g@addto@macro\UrlSpecials{\do\/{\Url@twoslashes}}
%\makeatother

%diagonal lines in a table - http://tex.stackexchange.com/questions/17745/diagonal-lines-in-table-cell
%slashbox is not available in texlive (due to licensing) and also gives bad results. This, we use diagbox
%\usepackage{diagbox}

%required for pdfcomment later
\usepackage[hyperref,svgnames]{xcolor}

\usepackage{listings}
\lstloadlanguages{java}
\lstset{language=java,numbers=left,captionpos=b}


%enable nice comments
%this also loads hyperref
\usepackage{pdfcomment}
%enable hyperref without colors and without bookmarks
\hypersetup{hidelinks,
   colorlinks=true,
   allcolors=black,
   pdfstartview=Fit,
   breaklinks=true}
%enables correct jumping to figures when referencing
\usepackage[all]{hypcap}

\newcommand{\commentontext}[2]{\colorbox{yellow!60}{#1}\pdfcomment[color={0.234 0.867 0.211},hoffset=-6pt,voffset=10pt,opacity=0.5]{#2}}
\newcommand{\commentatside}[1]{\pdfcomment[color={0.045 0.278 0.643},icon=Note]{#1}}

%compatibality with packages todo, easy-todo, todonotes
\newcommand{\todo}[1]{\commentatside{#1}}
%compatiblity with package fixmetodonotes
\newcommand{\TODO}[1]{\commentatside{#1}}

%enable \cref{...} and \Cref{...} instead of \ref: Type of reference included in the link

\usepackage[capitalise,nameinlink,ngerman]{cleveref}
%\usepackage[capitalise,nameinlink,english]{cleveref}
%Nice formats for \cref - only for English texts
%\crefname{section}{Sect.}{Sect.}
%\Crefname{section}{Section}{Sections}

\usepackage{xspace}
%\newcommand{\eg}{e.\,g.\xspace}
%\newcommand{\ie}{i.\,e.\xspace}
\newcommand{\eg}{e.\,g.,\ }
\newcommand{\ie}{i.\,e.,\ }

%introduce \powerset - hint by http://matheplanet.com/matheplanet/nuke/html/viewtopic.php?topic=136492&post_id=997377
\DeclareFontFamily{U}{MnSymbolC}{}
\DeclareSymbolFont{MnSyC}{U}{MnSymbolC}{m}{n}
\DeclareFontShape{U}{MnSymbolC}{m}{n}{
    <-6>  MnSymbolC5
   <6-7>  MnSymbolC6
   <7-8>  MnSymbolC7
   <8-9>  MnSymbolC8
   <9-10> MnSymbolC9
  <10-12> MnSymbolC10
  <12->   MnSymbolC12%
}{}
\DeclareMathSymbol{\powerset}{\mathord}{MnSyC}{180}

% correct bad hyphenation here
\hyphenation{op-tical net-works semi-conduc-tor}

%% END COPYING HERE


\begin{document}

\title{Paper Title}
%If Title is too long, use \titlerunning
%\titlerunning{Short Title}

\author{Vorname Nachname}

\supervisor{Vorname Nachname}

\seminar{Seminarbezeichung}

\semester{WS 2015/2016}

\abgabedatum{Stuttgart, 22.06.2015}

\institute{\email{autor.eins@beispiel.de}}

\frontpagede % creates the frontpage (in German)
%\frontpageen % creates the frontpage (in English)

\thispagestyle{empty}
\cleardoublepage

\maketitle

\begin{abstract}
Die Kurzfassung sollte den Inhalt der Arbeit kurz wiedergeben und aus mindestens 70 und maximal 150 Wörtern bestehen.
Sie wird in einer 9-Punkt Schrift gesetzt und 1 cm eingerückt von rechts und links.
Vor und nach der Kurzfassung sind zwei leere Zeilen.
\end{abstract}

%\begin{keywords}
%keyword1, keyword2
%\end{keywords}


\section{Einleitung}
Ab diesem Abschnitt fängt die eigentliche Arbeit an und man beginnt mit einer Einführung und der Einordnung des Themas.

Absätze werden durch Leerzeilen erreicht.

Als \LaTeX-Editor sind TeXstudio\footnote{\url{http://texstudio.sourceforge.net/}} oder TeXlipse\footnote{\url{http://texlipse.sourceforge.net/}} empfehlenswert.

Für die Versionskontrolle bietet es sich an, jeden Satz in einer neuen Zeile zu beginnen.

Als Nachschlagewerk und gute Einführung in das Thema \LaTeX\ bietet sich \emph{The Not So Short Introduction to \LaTeXe}\ von Oetiker et al an.
Auf der Seite \url{http://tobi.oetiker.ch/lshort/} ist dieses Dokument zum Herunterladen bereitgestellt.

Ein weiteres, empfehlenswertes Dokument ist das \emph{Kochbuch für \LaTeX{}}, welches auf der Seite \url{http://www.uni-giessen.de/hrz/tex/cookbook/cookbook.html} zu finden ist.

Eine Vorlage für Bachelor- und Masterarbeiten ist unter \url{https://github.com/latextemplates/uni-stuttgart-computer-science-template} zu finden.

Für weitere Anleitungen und Informationen empfiehlt sich die Homepage \url{http://www.dante.de/}.
Des Weiteren gibt es eine Reihe von Literatur zum Thema \LaTeX\ von Bürger~\cite{buerger}, Kopka~\cite{kopka}, Günther~\cite{guenther} und vielen anderen.
Weitere Hinweise zu \LaTeX{} unter \url{http://wiki.flupp.de/latex}.
Zu Ausarbeitungen im Allgemeinen unter \url{http://wiki.flupp.de/studium/ausarbeitungen}.

Die Arbeit gleidert sich wie folgt:
...
Schließlich wird in \cref{sec:zusfas} eine Zusammenfassung der Arbeit und ein Ausblick auf weitere Forschungsthemen gegeben.

\subsection{Beispiel für Überschrift 2.~Ebene}

\subsubsection{Beispiel für Überschrift 3.~Ebene}

\paragraph{Beispiel für Überschrift 4.~Ebene}
Dies ist Beispielabsatz zu einem Beispiel einer Überschrift 4.~Ebene.

\section{Formatierungsrichtlinien}

\subsection{Tabellen}
In diesem Abschnitt wird das Einfügen einer Tabelle und das Füllen derselben mit Werten gezeigt.
Im Gegensatz zu Abbildungen erfolgt die Beschriftung üblicherweise oberhalb und nicht unterhalb der Tabelle.
Der letzte Satz der Beschriftung endet ohne Punkt.
Die \Cref{tab:bsp} zeigt die Größenentwicklung der Weltbevölkerung von 8000 v.\,Chr. bis 1980 n.\,Chr.

\begin{table}
\caption{Dieses Beispiel entstammt dem {\it\TeX{}Buch,} S.\,246}
\label{tab:bsp}
\begin{center}
\begin{tabular}{r@{\quad}rl}
\hline
\multicolumn{1}{l}{\rule{0pt}{12pt}
                   Jahr}&\multicolumn{2}{l}{Weltbevölkerung}\\[2pt]
\hline\rule{0pt}{12pt}
8000 v.\,Chr. &     5,000,000& \\
  50 n\,Chr. &   200,000,000& \\
1650 n\,Chr. &   500,000,000& \\
1945 n\,Chr. & 2,300,000,000& \\
1980 n\,Chr. & 4,400,000,000& \\[2pt]
\hline
\end{tabular}
\end{center}
\end{table}

\subsection{Code-Beispiele}

\Cref{lst:example} zeigt ein Code-Beispiel.

\begin{lstlisting}[float,caption=A floating example,label=lst:example]
public static void main(String[] args) {
}
\end{lstlisting}

\subsection{Formeln}
Formeln sind in aufsteigender Reihenfolge, beginnend mit der Nummer~1, durchzunummerieren.
Dies erleichtert den Bezug vom Text auf die zugehörige Formel.
Die Formel
%
\begin{equation}
  \psi (u) = \int_{o}^{T} \left[\frac{1}{2}
  \left(\Lambda_{o}^{-1} u,u\right) + N^{\ast} (-u)\right] dt
\end{equation}
%
berechnet etwas.
Formeln sind bezüglich der Zeichensetzung wie normaler Text zu behandeln, d.\,h.\ falls ein Satz mit einer Formel endet, folgt dieser ein Punkt, der üblicherweise durch ein schmales Leerzeichen abgesetzt wird.

\subsection{Abbildungen}
Im nachfolgenden Beispiel wird dargestellt, wie Grafiken eingefügt werden und wie die Beschriftung bzw.\ die Formatierung derselben auszusehen hat.

Bei Abbildungen erfolgt die Beschriftung unterhalb der Abbildung (siehe \cref{fig:logo}).
Der letzte Satz der Beschreibung endet wie bei Tabellenbeschriftungen ohne Punkt.
Um eine gute Qualität zu erzielen, muss die Auflösung der Abbildungen mindestens 300~DPI betragen.
Bei Strichzeichnungen empfiehlt sich sogar eine weit höhere Auflösung, zwischen 800 und
1200~DPI.

Am besten Vektorgraphiken nehmen und als \texttt{.pdf} exportieren.

\begin{figure}
  \begin{center}
    \includegraphics[width=.5\textwidth]{ipvslogo.png}
    \caption{Das Logo des Instituts}
    \label{fig:logo}
   \end{center}
\end{figure}

\subsection{Literaturverweise}
Ein Verweis auf eine im Literaturverzeichnis aufgeführte
Literaturquelle wird durch die in eckige Klammern eingeschlossene
Nummer der entsprechenden Literaturquelle angegeben. Als Beispiel ist
hier nochmals auf Buerger~\cite{buerger} verwiesen.

\subsection{Anführungszeichen}
Entweder "`so"' tippen oder \enquote{so}.
``So'' bitte nicht, das führt zu englischen Anführungszeichen.

\subsection{Sonstiges}
Brackets work as designed:
<test>

The symbol for powerset is now correct: $\powerset$ and not a Weierstrass p ($\wp$).

\begin{inparaenum}
\item All these items...
\item ...appear in one line
\item This is enabled by the paralist package.
\end{inparaenum}

``something in quotes'' using plain tex or use \enquote{the enquote command}.

You can now write words containing hyphens which are hyphenated (application"=specific) at other places.
This is enabled by an additional configuration of the babel package.
In case you write \enquote{application-specific}, then the word will only be hyphenated at the dash.
You can also write applica\allowbreak{}tion-specific, but this is much more effort.

\section{Zusammenfassung}
\label{sec:zusfas}
Dieser Absatz schließt die Arbeit ab.

\subsubsection*{Acknowledgments}
...

In the bibliography, use \texttt{\textbackslash textsuperscript} for ``st'', ``nd'', ...:
E.g., \enquote{The 2\textsuperscript{nd} conference on examples}.
When you use \href{https://www.jabref.org}{JabRef}, you can use the clean up command to achieve that.
See \url{https://help.jabref.org/en/CleanupEntries} for an overview of the cleanup functionality.

%%%%%%%%%%%%%%%%%%%%%%%%%%%%%%%%%%%%%%%%%%%%%%%%%%%%%%%%%%%%%%%%%%%%%%%%%%%%%%%

\bibliographystyle{splncs03}
\bibliography{paper}
Alle Links wurden zuletzt am 22.06.2015 geprüft.
\end{document}
