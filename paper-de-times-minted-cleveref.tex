% Dieses Template wurde mit der "LLNCS DOCUMENT CLASS -- version 2.21 (12-Jan-2022)" getestet

% !TeX spellcheck = de-DE
% LTeX: language=de-DE
% !TeX encoding = utf8
% !TeX program = pdflatex
% !BIB program = bibtex
% -*- coding:utf-8 mod:LaTeX -*-

% Neue deutsche Trennmuster
% Siehe http://www.ctan.org/pkg/dehyph-exptl und http://projekte.dante.de/Trennmuster/WebHome
% Nur für pdflatex, nicht für lualatex
\RequirePackage[ngerman=ngerman-x-latest]{hyphsubst}

% "runningheads" zeigt Author + Titel auf jeder Seite.
% Diese Option nach Aufforderung durch die Herausgeber entfernen.
% Während des Schreibens und das Review des Papers hilft das, um z.B. auf konkrete Seitenzahlen einfach verweisen zu können.
\documentclass[runningheads,a4paper,ngerman]{llncs}[2022/01/12]

% backticks (`) werden als solches in verbatim-Umgebungen dargestellt
% Details unter:
%   - https://tex.stackexchange.com/a/341057/9075
%   - https://tex.stackexchange.com/a/47451/9075
%   - https://tex.stackexchange.com/a/166791/9075
\usepackage{upquote}

% Setze Deutsch als Sprache
\usepackage[english,main=ngerman]{babel}
%
% Hinweis von http://tex.stackexchange.com/a/321066/9075
% Ermögliche die Benutzung von "= als Trennstriche
\addto\extrasenglish{\languageshorthands{ngerman}\useshorthands{"}}
%
% Fix von https://tex.stackexchange.com/a/441701/9075
\usepackage{regexpatch}
\makeatletter
\edef\switcht@albion{%
  \relax\unexpanded\expandafter{\switcht@albion}%
}
\xpatchcmd*{\switcht@albion}{ \def}{\def}{}{}
\xpatchcmd{\switcht@albion}{\relax}{}{}{}
\edef\switcht@deutsch{%
  \relax\unexpanded\expandafter{\switcht@deutsch}%
}
\xpatchcmd*{\switcht@deutsch}{ \def}{\def}{}{}
\xpatchcmd{\switcht@deutsch}{\relax}{}{}{}
\edef\switcht@francais{%
  \relax\unexpanded\expandafter{\switcht@francais}%
}
\xpatchcmd*{\switcht@francais}{ \def}{\def}{}{}
\xpatchcmd{\switcht@francais}{\relax}{}{}{}
\makeatother

% Links verhalten sich so, wie sie sollen
% Zeilenumbrüche bei URLs auch bei Bindestrichen erlauben, auch wenn es verwirrend sein könnte: Gehört der Bindestrich zur URL oder ist es ein Trennstrich?
% Siehe https://tex.stackexchange.com/a/3034/9075.
\usepackage[hyphens]{url}
% \urlstyle{same}
%
% Hinweis von http://tex.stackexchange.com/a/10419/9075.
\makeatletter
\g@addto@macro{\UrlBreaks}{\UrlOrds}
\makeatother

\RequirePackage{newtxtext}
\RequirePackage{newtxmath}
\RequirePackage[zerostyle=b,scaled=.9]{newtxtt}

% Has to be loaded AFTER any font packages. See https://tex.stackexchange.com/a/2869/9075.
\usepackage[T1]{fontenc}

% Optischer Randausgleich und Grauwertkorrektur. Siehe See http://www.ctan.org/tex-archive/macros/latex/contrib/microtype/

\usepackage[
  babel=true,
  expansion=alltext,
  protrusion=alltext-nott,
  final
]{microtype}

% \texttt{test -- test} - diese Einstellung behält "--" bei (und konveriert sie nicht zu einem Bindestrich)
\DisableLigatures{encoding = T1, family = tt* }

% tracking=true muss als Parameter des microtype-packages mitgegeben werden
% Deaktiviert, da dies bei Algorithmen seltsam aussieht

%\DeclareMicrotypeSet*[tracking]{my}{ font = */*/*/sc/* }%

% Hier wird festgelegt, dass alle Passagen in Kapitälchen automatisch leicht gesperrt werden.
% Quelle: http://homepage.ruhr-uni-bochum.de/Georg.Verweyen/pakete.html
% Deaktiviert, da sonst "BPEL", "BPMN" usw. wirklich komisch aussehen.
% Macht wohl nur bei geisteswissenschaftlichen Arbeiten Sinn.
%\SetTracking{ encoding = *, shape = sc }{ 45 }

\usepackage{graphicx}

% Diagonal lines in a table - http://tex.stackexchange.com/questions/17745/diagonal-lines-in-table-cell
% Slashbox is not available in texlive (due to licensing) and also gives bad results. Thus, we use diagbox
\usepackage{diagbox}

\usepackage{xcolor}

\usepackage[newfloat]{minted}

\setminted{
    % Line numbers not flowing out of the margin
    numbersep=5pt,
    xleftmargin=12pt,
    %
    % Better listing breaking
    breakafter=-/\{,
    breakbefore=\\
    %
    % Alternative: Rely on pygment's tokenizer. Does not work well with LaTeX and comments
    % breakbytoken=true,
    % breakbytokenanywhere=true
}

\usemintedstyle{bw} %black and white style
%\usemintedstyle{vs} %visual studio
%\usemintedstyle{friendlygrayscale} % custom style - submitted as pull request https://bitbucket.org/birkenfeld/pygments-main/pull-requests/748/add-style-friendly-grayscale/diff
%\usemintedstyle{friendly}
%\usemintedstyle{eclipse} %http://www.jevon.org/wiki/Eclipse_Pygments_Style
%\usemintedstyle{autumn}
%\usemintedstyle{rrt}
%\usemintedstyle{borland}

% We need to load caption to have a bold font on the label
% The other parameters mimic the layout of the LNCS class
\usepackage[labelfont=bf,font=small,skip=4pt]{caption}
\SetupFloatingEnvironment{listing}{name=List.,within=none}

% When using both minted and listings
% Compatibility of packages minted and listings with respect to the numbering of "List." caption
% Source: https://tex.stackexchange.com/a/269510/9075
% \AtBeginEnvironment{listing}{\setcounter{listing}{\value{lstlisting}}}
% \AtEndEnvironment{listing}{\stepcounter{lstlisting}}

% Intermediate solution for hyperlinked refs. See https://tex.stackexchange.com/q/132420/9075 for more information.
\newcommand{\labelline}[1]{\label[line]{#1}\hypertarget{#1}{}}
\newcommand{\refline}[1]{\hyperlink{#1}{\FancyVerbLineautorefname~\ref*{#1}}}


% For easy quotations: \enquote{text}
% This package is very smart when nesting is applied, otherwise textcmds (see below) provides a shorter command
\usepackage[autostyle=true]{csquotes}

% Enable using "`quote"' - see https://tex.stackexchange.com/a/150954/9075
\defineshorthand{"`}{\openautoquote}
\defineshorthand{"'}{\closeautoquote}

% Nicer tables (\toprule, \midrule, \bottomrule)
\usepackage{booktabs}

% Extended enumerate, such as \begin{compactenum}
\usepackage{paralist}

% Put figures aside a text
\usepackage[rflt]{floatflt}

% Bibliopgraphy enhancements
%  - enable \cite[prenote][]{ref}
%  - enable \cite{ref1,ref2}
% Alternative: \usepackage{cite}, which enables \cite{ref1, ref2} only (otherwise: Error message: "White space in argument")

% Doc: http://texdoc.net/natbib
\usepackage[%
  square,        % for square brackets
  comma,         % use commas as separators
  numbers,       % for numerical citations;
%  sort,          % orders multiple citations into the sequence in which they appear in the list of references;
  sort&compress, % as sort but in addition multiple numerical citations
                 % are compressed if possible (as 3-6, 15);
]{natbib}
% In the bibliography, references have to be formatted as 1., 2., ... not [1], [2], ...
\renewcommand{\bibnumfmt}[1]{#1.}

% Prepare more space-saving rendering of the bibliography
% Source: https://tex.stackexchange.com/a/280936/9075
\SetExpansion
[ context = sloppy,
  stretch = 30,
  shrink = 60,
  step = 5 ]
{ encoding = {OT1,T1,TS1} }
{ }

% Enable nice comments
\usepackage{pdfcomment}

\newcommand{\commentontext}[2]{\colorbox{yellow!60}{#1}\pdfcomment[color={0.234 0.867 0.211},hoffset=-6pt,voffset=10pt,opacity=0.5]{#2}}
\newcommand{\commentatside}[1]{\pdfcomment[color={0.045 0.278 0.643},icon=Note]{#1}}

% Compatibality with packages todo, easy-todo, todonotes
\newcommand{\todo}[1]{\commentatside{#1}}

% Compatiblity with package fixmetodonotes
\newcommand{\TODO}[1]{\commentatside{#1}}


% Fußnoten unter Gleitumgebungen ("floats") platzieren
% Quelle: https://tex.stackexchange.com/a/32993/9075
\usepackage{stfloats}
\fnbelowfloat

\usepackage[group-minimum-digits=4,per-mode=fraction]{siunitx}
\addto\extrasgerman{\sisetup{locale = DE}}

% Enable that parameters of \cref{}, \ref{}, \cite{}, ... are linked so that a reader can click on the number an jump to the target in the document
\usepackage{hyperref}

% Enable hyperref without colors and without bookmarks
\hypersetup{
  hidelinks,
  colorlinks=true,
  allcolors=black,
  pdfstartview=Fit,
  breaklinks=true
}

% Enable correct jumping to figures when referencing
\usepackage[all]{hypcap}

% Hint by https://tex.stackexchange.com/a/193370/9075 to suppress strange outputs of the babel package
% Example strange output: Package babel Info: Redefining ngerman shorthand "|
\usepackage{etoolbox}
\makeatletter
\patchcmd{\@decl@short}{\bbl@info}{\@gobble}{}{}
\makeatother

\usepackage{mindflow}
% cleveref für cref statt autoref, da cleveref auch bei Definitionen funktioniert
\usepackage[capitalise,nameinlink]{cleveref}

\crefname{table}{Tabelle}{Tab.}
\Crefname{table}{Tabelle}{Tabellen}
\crefname{figure}{Abbildung}{Abbildungen}
\Crefname{figure}{Abbildung}{Abbildungen}
\crefname{equation}{Gleichung}{Gleichungen}
\Crefname{equation}{Gleichung}{Gleichungen}
\crefname{theorem}{Theorem}{Theoreme}
\Crefname{theorem}{Theorem}{Theoreme}
\crefname{listing}{Listing}{Listings}
\Crefname{listing}{Listing}{Listings}
\crefname{section}{Abschnitt}{Abschnitte}
\Crefname{section}{Abschnitt}{Abschnitte}
\crefname{paragraph}{Abschnitt}{Abschnitte}
\Crefname{paragraph}{Abschnitt}{Abschnitte}
\crefname{subparagraph}{Abschnitt}{Abschnitte}
\Crefname{subparagraph}{Abschnitt}{Abschnitte}

% Intermediate solution for hyperlinked refs. See https://tex.stackexchange.com/q/132420/9075 for more information.
\newcommand{\llabel}[1]{\label[line]{#1}\hypertarget{#1}{}}
\newcommand{\lref}[1]{\hyperlink{#1}{\FancyVerbLineautorefname~\ref*{#1}}}


\usepackage{lipsum}

% For demonstration purposes only
% These packages can be removed when all examples have been deleted
\usepackage[math]{blindtext}
\usepackage{mwe}
\usepackage[realmainfile]{currfile}
\usepackage{tcolorbox}
\tcbuselibrary{minted}

%introduce \powerset - hint by http://matheplanet.com/matheplanet/nuke/html/viewtopic.php?topic=136492&post_id=997377
\DeclareFontFamily{U}{MnSymbolC}{}
\DeclareSymbolFont{MnSyC}{U}{MnSymbolC}{m}{n}
\DeclareFontShape{U}{MnSymbolC}{m}{n}{
  <-6>    MnSymbolC5
  <6-7>   MnSymbolC6
  <7-8>   MnSymbolC7
  <8-9>   MnSymbolC8
  <9-10>  MnSymbolC9
  <10-12> MnSymbolC10
  <12->   MnSymbolC12%
}{}
\DeclareMathSymbol{\powerset}{\mathord}{MnSyC}{180}


% Enable hyphenation at other places as the dash.
% Example: applicaiton\hydash specific
\makeatletter
\newcommand{\hydash}{\penalty\@M-\hskip\z@skip}
% Definition of "= taken from http://mirror.ctan.org/macros/latex/contrib/babel-contrib/german/ngermanb.dtx
\makeatother

% correct bad hyphenation here
\hyphenation{op-tical net-works semi-conduc-tor}

% Add copyright
% Do that for the final version or if you send it to colleagues
\iffalse
  %state: intended|submitted|llncs
  %you can add "crop" if the paper should be cropped to the format Springer is publishing
  \usepackage[intended]{llncsconf}

  \conference{name of the conference}

  %in case of "llncs" (final version!)
  %example: llncs{Anonymous et al. (eds). \emph{Proceedings of the International Conference on \LaTeX-Hacks}, LNCS~42. Some Publisher, 2016.}{0042}
  \llncs{book editors and title}{0042} %% 0042 is the start page
\fi

% Enable copy and paste of text from the PDF
% Only required for pdflatex. It "just works" in the case of lualatex.
% Alternative: cmap or mmap package
% mmap enables mathematical symbols, but does not work with the newtx font set
% See: https://tex.stackexchange.com/a/64457/9075
% Other solutions outlined at http://goemonx.blogspot.de/2012/01/pdflatex-ligaturen-und-copynpaste.html and http://tex.stackexchange.com/questions/4397/make-ligatures-in-linux-libertine-copyable-and-searchable
% Trouble shooting outlined at https://tex.stackexchange.com/a/100618/9075
%
% According to https://tex.stackexchange.com/q/451235/9075 this is the way to go
\input glyphtounicode
\pdfgentounicode=1

\begin{document}

\title{Paper Title}
% If Title is too long, use \titlerunning
%\titlerunning{Short Title}

% Single insitute
\author{Firstname Lastname \and Firstname Lastname}

% If there are too many authors, use \authorrunning
%\authorrunning{First Author et al.}

\institute{Institute}

%% Multiple insitutes - ALTERNATIVE to the above
% \author{%
%     Firstname Lastname\inst{1} \and
%     Firstname Lastname\inst{2}
% }
%
%If there are too many authors, use \authorrunning
%  \authorrunning{First Author et al.}
%
%  \institute{
%      Insitute 1\\
%      \email{...}\and
%      Insitute 2\\
%      \email{...}
%}

\maketitle

\begin{abstract}
  \lipsum[1]
\end{abstract}

\begin{keywords}
  keyword1, keyword2
\end{keywords}

\section{Introduction}
\label{sec:intro}
\lipsum[1-3]\todo{Refine me}

The remainder of the paper starts with a presentation of related work (\cref{sec:relatedwork}).
It is followed by a presentation of hints on \LaTeX{} (\cref{sec:latexhints}).
Finally, a conclusion is drawn and outlook on future work is made (\cref{sec:outlook}).

\section{Related Work}
\label{sec:relatedwork}

Winery~\cite{Winery} is a graphical \commentontext{modeling}{modeling with one ``l'', because of AE} tool.
The whole idea of TOSCA is explained by \citet{Binz2009}.

\section{LaTeX Hinweise}
\label{chap:latexhints}

% Benötigt für eine korrekte Darstellung der Hinweise im erzeugten PDF
\newcount\LTGbeginlineexample
\newcount\LTGendlineexample
\newenvironment{ltgexample}%
{\LTGbeginlineexample=\numexpr\inputlineno+1\relax}%
{%
\LTGendlineexample=\numexpr\inputlineno-1\relax%

\tcbinputlisting{%
  listing only,
  listing file=\currfilepath,
  colback=green!5!white,
  colframe=green!25,
  coltitle=black!90,
  coltext=black!90,
  left=8mm,
  title=Zugehöriger \LaTeX{}-Quelltext aus \texttt{\currfilepath},
    minted language=TeX,
    minted style=vs,
    minted options={
      fontsize=\footnotesize,
      firstline=\the\LTGbeginlineexample,
      lastline=\the\LTGendlineexample,
      firstnumber=\the\LTGbeginlineexample,
      breaklines,
      linenos,
      numbersep=8pt
    }
}
}%

Hier sollen allgemeine \LaTeX-Hinweise gegeben werden, damit man Minimalbeispiele vorliegen hat, um sofort loszulegen.

\subsection{Trennung von Absätzen}

\begin{ltgexample}
Pro Satz eine neue Zeile.
Das ist wichtig, um sauber versionieren zu können.
In LaTeX werden Absätze durch eine Leerzeile getrennt.
Analogie zu Word: Bei Word werden neue Absätze durch einmal Eingabetaste herbeigeführt.
Dies führt bei LaTeX jedoch nicht zu einem neuen Absatz, da LaTeX direkt aufeinanderfolgende Zeilen zu einer Zeile zusammenfügt.
Mächte man nun einen Absatz haben, muss man zweimal die Eingabetaste drücken.
Dies führt zu einer leeren Zeile.
In Word gibt es die Funktion Großschreibetaste und Eingabetaste gleichzeitig.
Wenn man dies drückt, wird einer harter Umbruch erzwungen.
Der Text fängt am Anfang der neuen Zeile an.
In LaTeX erreicht man dies durch Doppelbackslashes (\textbackslash\textbackslash) erzeugt.\\
Dies verwendet man quasi nie.

Folglich werden neue Abstäze insbesondere \emph{nicht} durch Doppelbackslashes erzeugt.
Beispielsweise begann der letzte Satz in einem neuen Absatz.
Eine ausführliche Motivation hierfür findet sich in \url{http://loopspace.mathforge.org/HowDidIDoThat/TeX/VCS/#section.3}.
\end{ltgexample}


\subsection{Notes separated from the text}

The package mindflow enables writing down notes and annotations in a way so that they are separated from the main text.

\begin{ltgexample}
\begin{mindflow}
This is a small note.
\end{mindflow}
\end{ltgexample}
\subsection{Hyphenation}

\LaTeX{} automatically hyphenates words.
When using microtype, there should be less hypnetations than in other settings.
It might be necessary to tweak the hyphenations nevertheless.
Here are some hints:

\begin{ltgexample}
In case you write \enquote{application-specific}, then the word will only be hyphenated at the dash.
You can also write \verb1applica\allowbreak{}tion-specific1 (result: applica\allowbreak{}tion-specific), but this is much more effort.

You can now write words containing hyphens which are hyphenated at other places in the word.
For instance, \verb1application"=specific1 gets application"=specific.
This is enabled by an additional configuration of the babel package.
\end{ltgexample}

\subsection{Typesetting Units}

\begin{ltgexample}
Numbers can written plain text (such as 100), by using the siunitx package like that:
\SI{100}{\km\per\hour},
or by using plain \LaTeX{} (and math mode):
$100 \frac{\mathit{km}}{h}$.
\end{ltgexample}

\begin{ltgexample}
\SI{5}{\percent} of \SI{10}{kg}
\end{ltgexample}

\begin{ltgexample}
Numbers are automatically grouped: \num{123456}.
\end{ltgexample}

\subsection{Surrounding Text by Quotes}

\begin{ltgexample}
Please use the \enquote{enquote command} to quote something.
Quoting with "`quote"' or ``quote'' also works.

\end{ltgexample}
\subsection{Cleveref examples}
\label{sec:ex:cref}

Cleveref demonstration: Cref at beginning of sentence, cref in all other cases.

\begin{figure}
    \centering
    \includegraphics[width=.75\linewidth]{example-image-a}
    \caption{Example figure for cref demo}
    \label{fig:ex:cref}
\end{figure}

\begin{table}
    \centering
    \begin{tabular}{ll}
      \toprule
      Heading1 & Heading2 \\
      \midrule
      One      & Two      \\
      Thee     & Four     \\
      \bottomrule
    \end{tabular}
    \caption{Example table for cref demo}
    \label{tab:ex:cref}
\end{table}

\begin{ltgexample}
\Cref{fig:ex:cref} shows a simple fact, although \cref{fig:ex:cref} could also show something else.

\Cref{tab:ex:cref} shows a simple fact, although \cref{tab:ex:cref} could also show something else.

\Cref{sec:ex:cref} shows a simple fact, although \cref{sec:ex:cref} could also show something else.
\end{ltgexample}

\subsection{Abbildungen}

\begin{ltgexample}
\Cref{fig:label} zeigt etwas Interessantes

\begin{figure}
  \centering
  Füge deine Abbildung hier ein.
  \caption{Bildunterschrift.}
  \label{fig:label}
\end{figure}
\end{ltgexample}

One can also have pictures floating inside text:
\clearpage

\begin{ltgexample}
\begin{floatingfigure}{.33\linewidth}
\includegraphics[width=.29\linewidth]{example-image-a}
\caption{A floating figure}
\end{floatingfigure}
\blindtext[2]
\end{ltgexample}


\subsection{Tables}

\begin{ltgexample}
\begin{table}
  \caption{Simple Table}
  \label{tab:simple}
  \centering
  \begin{tabular}{ll}
    \toprule
    Heading1 & Heading2 \\
    \midrule
    One      & Two      \\
    Thee     & Four     \\
    \bottomrule
  \end{tabular}
\end{table}
\end{ltgexample}

\begin{ltgexample}
% Source: https://tex.stackexchange.com/a/468994/9075
\begin{table}
\caption{Table with diagonal line}
\label{tab:diag}
\begin{center}
\begin{tabular}{|l|c|c|}
\hline
\diagbox[width=10em]{Diag\\Column Head I}{Diag Column\\Head II} & Second & Third \\
\hline
& foo & bar \\
\hline
\end{tabular}
\end{center}
\end{table}
\end{ltgexample}


\subsection{Source Code}

\href{https://github.com/gpoore/minted}{minted} is a sophisticated package to enable properly highlighted listings.
It uses the \href{http://pygments.org/}{pygments} library, which in turn requires Python.

\begin{ltgexample}
\Cref{lst:XML} shows source code written in XML.
\refline{line:comment} contains a comment.

\begin{listing}[htbp]
    \begin{minted}[linenos=true,escapeinside=||]{xml}
<listing name="example">
  <!-- comment --> |\labelline{line:comment}|
  <content>not interesting</content>
</listing>
\end{minted}
  \caption{Example XML listing using minted}
  \label{lst:XML}
\end{listing}
\end{ltgexample}

One can also typeset JSON as shown in \cref{lst:flJSON}.

\begin{ltgexample}
\begin{listing}[htbp]
    \begin{minted}[linenos=true,escapeinside=||]{json}
{
  key: "value"
}
\end{minted}
  \caption{Example JSON listing using minted}
  \label{lst:flJSON}
\end{listing}
\end{ltgexample}

Java is also possible as shown in \cref{lst:flJava}.

\begin{ltgexample}
\begin{listing}[htbp]
    \begin{minted}[linenos=true,escapeinside=||]{java}
public class Hello {
    public static void main (String[] args) {
        System.out.println("Hello World!");
    }
}
\end{minted}
  \caption{Java code rendered using minted}
  \label{lst:flJava}
\end{listing}
\end{ltgexample}

\subsection{Itemization}

One can list items as follows:

\begin{ltgexample}
\begin{itemize}
\item Item One
\item Item Two
\end{itemize}
\end{ltgexample}


One can enumerate items as follows:

\begin{ltgexample}
\begin{enumerate}
  \item Item One
  \item Item Two
\end{enumerate}
\end{ltgexample}


With paralist, one can even have all items typset after each other and have them clean in the tex document:

\begin{ltgexample}
\begin{inparaenum}
  \item All these items...
  \item ...appear in one line
  \item This is enabled by the paralist package.
\end{inparaenum}
\end{ltgexample}

\subsection{Other Features}

\begin{ltgexample}
The words \enquote{workflow} and \enquote{dwarflike} can be copied from the PDF and pasted to a text file.
\end{ltgexample}

\begin{ltgexample}
The symbol for powerset is now correct: $\powerset$ and not a Weierstrass p ($\wp$).

$\powerset({1,2,3})$
\end{ltgexample}

\begin{ltgexample}
Brackets work as designed:
<test>
One can also input backquotes in verbatim text: \verb|`test`|.
\end{ltgexample}


\section{Zusammenfassung und Ausblick}
\label{sec:outlook}
\lipsum[1-2]

\subsubsection*{Danksagungen}
\ldots

In the bibliography, use \texttt{\textbackslash textsuperscript} for \enquote{st}, \enquote{nd}, \ldots:
E.g., \enquote{The 2\textsuperscript{nd} conference on examples}.
When you use \href{https://www.jabref.org}{JabRef}, you can use the clean up command to achieve that.
See \url{https://help.jabref.org/en/CleanupEntries} for an overview of the cleanup functionality.

\renewcommand{\bibsection}{\section*{Literatur}} % requried for natbib to have "References" printed and as section*, not chapter*
% Use natbib compatbile splncsnat style.
% It does provide all features of splncs03, but is developed in a clean way.
% Source: http://phaseportrait.blogspot.de/2011/02/natbib-compatible-bibtex-style-bst-file.html
\bibliographystyle{splncsnat}
\begingroup
  \microtypecontext{expansion=sloppy}
  \small % ensure correct font size for the bibliography
  \bibliography{paper}
\endgroup

% Enfore empty line after bibliography
\ \\
%
Alle Links wurden zuletzt am 29.03.2021 geprüft.
\end{document}
